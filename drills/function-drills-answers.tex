\documentclass{article}
\usepackage{amsbsy,amscd,amsfonts,amsgen,amsmath,amsopn,amssymb,amstext,amsthm,amsxtra}
\usepackage{graphicx}
\usepackage{verbatim}
\usepackage{multicol}
\usepackage{fullpage}
\begin{document}

\title{Function Drills with Answers}
\author{Statistics 405}
\date{\today}
\maketitle

%-------------------------------------------------------------------
% Begin entering your LaTeX document content here...

\noindent Write a simple function in R for each of the following tasks. 
\begin{enumerate}

\item Return the circumference of a circle with the given radius.
  \begin{quote}
    \begin{verbatim}
c_circ <- function(r){
  2 * pi * r
}    
    \end{verbatim}
  \end{quote}

\item Return the area of a circle with the given radius.
  \begin{quote}
    \begin{verbatim}
c_area <- function(r){
  pi * r ^ 2
}
    \end{verbatim}
  \end{quote}
  
  \item Return the area of a circle with the given radius.
  \begin{quote}
    \begin{verbatim}
c_vol <- function(r){
  4 / 3 * pi * r ^ 3
}	    
    \end{verbatim}
  \end{quote}
  
\item Return the circumference, area (of the largest cross-section), and volume of a sphere with the given radius. Each should be labelled in the functions output.
  \begin{quote}
    \begin{verbatim}
c_stats <- function(r){
  c(circumference = c_circ(r),
    area = c_area(r),
    volume = c_vol(r)
  )
}    
    \end{verbatim}
  \end{quote}

\item Given the coefficients of a quadratic polynomial, return the roots.
  \begin{quote}
    \begin{verbatim}
quad_formula <- function(a, b, c){
  stopifnot(b ^ 2 >= 4 * a * c)
  c((-b - sqrt(b ^ 2 - 4 * a * c)) / (2 * a),
    (-b + sqrt(b ^ 2 - 4 * a * c)) / (2 * a))
}    
    \end{verbatim}
  \end{quote}

\item Return the lowest positive value of a vector
  \begin{quote}
    \begin{verbatim}
min_pos <- function(vec){
  pos <- vec[vec > 0]
  min(pos)
}    
    \end{verbatim}
  \end{quote}

\item Return the second lowest positive value of a vector
  \begin{quote}
    \begin{verbatim}
min2_pos <- function(vec){
  pos <- vec[vec > 0]
  pos <- pos[-which(pos == min(pos))]
  min(pos)
}
    \end{verbatim}
  \end{quote}


\item Divide each element in a numeric vector by the vector's length.
  \begin{quote}
    \begin{verbatim}
div_vec <- function(vec){
  vec / length(vec)
}
    \end{verbatim}
  \end{quote}


\item Test whether a number is even.
  \begin{quote}
    \begin{verbatim}
is.even <- function(num){
  num %% 2 == 0
}

    \end{verbatim}
  \end{quote}
  
\item Test whether a number is odd.
  \begin{quote}
    \begin{verbatim}
is.odd <- function(num){
  num %% 2 == 1
}
    \end{verbatim}
  \end{quote}

\item If a number is odd add one to it.
  \begin{quote}
    \begin{verbatim}
one_to_odd <- function(num){
  if (is.odd(num))
    return(num + 1)
  return(num)
} 
    \end{verbatim}
  \end{quote}

\item If any number in a numeric vector is odd, add one to it
  \begin{quote}
    \begin{verbatim}
one_to_odd2 <- function(vec){
  vec[is.odd(vec)] <- vec[is.odd(vec)] + 1
  vec
}	

# note: this function is better than the previous one
# It works for numbers AND vectors, and it is simpler
    \end{verbatim}
  \end{quote}

\item Test whether a number is an integer.
  \begin{quote}
    \begin{verbatim}
integer <- function(a){
  trunc(a) == a
}
    \end{verbatim}
  \end{quote}


\item Find the range of a vector.
  \begin{quote}
    \begin{verbatim}
my_range <- function(vec){
  max(vec) - min(vec)
}
    \end{verbatim}
  \end{quote}

\item Find the sum of a numeric vector (without using \verb!sum()!).
  \begin{quote}
     \begin{verbatim}
	my_sum <- function(vec){
	  total <- 0
	  for (i in 1:length(vec)){
	  	total <- total + vec[i]
	  }
	  total
	}
    \end{verbatim}
  \end{quote}

\item Find the mean of a numeric vector (without using \verb!mean()!).
  \begin{quote}
    \begin{verbatim}
my_mean <- function(vec){
  my_sum(vec)/length(vec)
}
    \end{verbatim}
  \end{quote}
  
\item Find the mean of a vector that contains one or more NA's by ignoring any NA's (without using \verb!mean()!).
  \begin{quote}
    \begin{verbatim}
my_mean2 <- function(vec){
  my_mean(na.omit(vec))
}

    \end{verbatim}
  \end{quote}

\item Find the variance of a numeric vector (without using \verb!var()!).
  \begin{quote}
    \begin{verbatim}
my_var <- function(vec){
  xbar <- my_mean(vec)
  n <- length(vec)
  sum <- 0
  
  for (i in 1:n){
  	sum <- sum + (vec[i] - xbar) ^ 2
  }
  
  # let's use n-1 so we can compare with R's var(). 
  # This is the sample variance
  sum / (n - 1)
 }	

    \end{verbatim}
  \end{quote}



\item Automatically create a histogram of a vector.
  \begin{quote}
    \begin{verbatim}
my_hist <- function(vec){
  hist(vec)
}
    \end{verbatim}
  \end{quote}

\item Automatically create a histogram of a vector with the given number of bins.
  \begin{quote}
    \begin{verbatim}
better_hist <- function(vec, n){
  hist(vec, breaks = n)
}
    \end{verbatim}
  \end{quote}
  
\item Automatically create a scatterplot matrix with the variables in a given data frame.
  \begin{quote}
    \begin{verbatim}
scatterplot <- function(df){
  library(ggplot2)
  plotmatrix(df)
}
    \end{verbatim}
  \end{quote}
  

\item Find the least common multiple of two numbers.
  \begin{quote}
    \begin{verbatim}
LCM <- function(a,b){
  test <- a * c(1:b)
  multiples <- test[integer(test/b)] 
  min(multiples)
} 
    \end{verbatim}
  \end{quote}

\item Index a series of observations by the first observation (hint: express each observation as a percentage of the first observation).
  \begin{quote}
    \begin{verbatim}
how_to_index <- function(vec){
  vec / vec[1] * 100
}     
    \end{verbatim}
  \end{quote}

\item Find the determinant of a four by four matrix.
  \begin{quote}
    \begin{verbatim}
determinant <- function(matrix){
  matrix[1,1] * matrix[2,2] - matrix[1,2] * matrix[2,1]
}    
    \end{verbatim}
  \end{quote}


  
\item Separate the integer and decimal parts of a number, return them in a vector of length two.
  \begin{quote}
    \begin{verbatim}
split_num <- function(num){
  c(trunc(num), num - trunc(num))
}
    \end{verbatim}
  \end{quote}

\item Return the given vector with all NA's removed.
  \begin{quote}
    \begin{verbatim}
clean <- function(vec){
  na.omit(vec)
}
    \end{verbatim}
  \end{quote}



\item Return the row numbers of rows in a data frame that contain NA's.
  \begin{quote}
    \begin{verbatim}
get_NAs <- function(df){
  new <- na.omit(df)
  setdiff(row.names(df), row.names(new))
}
    \end{verbatim}
  \end{quote}
  
\item Return the actual rows of a data frame that contain NA's.
  \begin{quote}
    \begin{verbatim}
get_NAs2 <- function(df){
  new <- na.omit(df)
  rows <- setdiff(row.names(df), row.names(new))
  df[rows,]
}
    \end{verbatim}
  \end{quote}
  

\item Create a new vector by repeating a given vector a given number of times.
  \begin{quote}
    \begin{verbatim}
rep_vec <- function(vec, n){
  rep(vec, n)	
}
    \end{verbatim}
  \end{quote}

\item Double each element in a vector (e.g., turn \emph{a,b,c,...} into \emph{a, a, b, b, c,...}).
  \begin{quote}
    \begin{verbatim}
rep_vec2 <- function(vec, n){
  vec[rep(1:length(vec), each = n)]
}
    \end{verbatim}
  \end{quote}


\item Randomly return one of the following phrases, ``Ace", ``King" or ``Queen" with equal probability of returning each.
  \begin{quote}
    \begin{verbatim}
random <- function(){
  sample(c("Ace", "King", "Queen"), 1)
}
    \end{verbatim}
  \end{quote}

\item Randomly return one of the following phrases, ``Ace", ``King" or ``Queen" with twice as much probability of returning ``Ace" as either ``King" or ``Queen."
  \begin{quote}
    \begin{verbatim}
 random2 <- function(){
  sample(c("Ace", "King", "Queen"), 1, prob = c(2, 1, 1))
}
    \end{verbatim}
  \end{quote}

\item Take any character string and add ``...in Stat 405" to the end.
  \begin{quote}
    \begin{verbatim}
fortune_cookie <- function(fortune){
  paste(fortune, "...in Stat405.")
}
    \end{verbatim}
  \end{quote}

\item Take any character string and add ``...in Stat 405" to the end. Check that the input is a character string. Return an error if it is not.
  \begin{quote}
    \begin{verbatim}
fortune_cookie <- function(fortune){
  stopifnot(is.character(fortune))
  paste(fortune, "...in Stat405.")
}	
    \end{verbatim}
  \end{quote}


\item Save the current graph with width = 6 and height = 6 as a pdf with the inputted name.
  \begin{quote}
    \begin{verbatim}
save_plot <- function(name){
  filename <- paste("name", "pdf", sep = ".")
  ggsave(filename, width = 6, height = 6)
}    
    \end{verbatim}
  \end{quote}

\item Save the current graph with a given width and height as a pdf with the inputted name.
  \begin{quote}
    \begin{verbatim}
save_plot2 <- function(name, width, height){
  filename <- paste("name", "pdf", sep = ".")
  ggsave(filename, width = width, height = height)
}    
    \end{verbatim}
  \end{quote}
  
\item Save a copy of a data frame as a comma separated values file whose filename is the name of the data frame plus ".csv"
  \begin{quote}
    \begin{verbatim}
save_file <- function(df){
  filename <- paste(substitute(df), "csv", sep = ".")

  # best method to avoid adding row numbers
  write.table(file, filename, sep = ",", row = F)
}

    \end{verbatim}
  \end{quote}

\item Identify whether an object is a logical, character, or numeric object.
  \begin{quote}
    \begin{verbatim}
type <- function(obj){
  mode(obj)
}
    \end{verbatim}
  \end{quote}
  
\item Display the number of groups of size n can be made from the inputted vector of length k.
  \begin{quote}
    \begin{verbatim}
n_choose_k <- function(n, k){
  factorial(n) / (factorial(k) * factorial(n - k))
}
    
    \end{verbatim}
  \end{quote}

\item Return the number of unique permutations that can be from a given vector (caution: don't use large vectors).
  \begin{quote}
    \begin{verbatim}
num_seqs <- function(vec){
  n <- length(vec)
  factorial(n)
}    
    \end{verbatim}
  \end{quote}

\item Return the number of unique sets that can be made from an inputted vector.
  \begin{quote}
    \begin{verbatim}
num_sets <- function(vec){
  vec <- unique(vec)
  n <- length(vec)
  my_sum(2 ^ c(0:(n-1)))
}
    \end{verbatim}
  \end{quote}

\item Return whichever the entered number is closest to: 0 or 1000.
  \begin{quote}
    \begin{verbatim}
closest <- function(a){
  if (a > 1000){
  	a <- 1000
  }
  
  round(a, -3)
}
    \end{verbatim}
  \end{quote}

\item Given a data frame with two columns, return all of the combinations of the two variables that occur once or more.
  \begin{quote}
    \begin{verbatim}
combos <- function(df){
  combinations <- table(df[,1], df[,2])
  df_counts <- as.data.frame(combinations)
  names(df_counts) <- c(names(df), "count")
  df_counts <- subset(df_counts, count > 0)
  df_counts
}    
    \end{verbatim}
  \end{quote}

\item Automatically plot the above results with each variable on an axis and the number of occurrences (counts) represented by color.
  \begin{quote}
    \begin{verbatim}
colorful_counts <- function(df){
  df_counts <- combos(df)
  df_counts <- within(df_counts, {
    x <- as.numeric(as.character(df_counts[,1]))
    y <- as.numeric(as.character(df_counts[,1]))
  })
  qplot(x, y, data = df_counts, colour = count)
}     
    \end{verbatim}
  \end{quote}

\item Create a new vector where each \emph{ith} element is the sum of the first \emph{i} elements of the given vector.
  \begin{quote}
    \begin{verbatim}
cumsum <- function(vec){
  new <- vector(length = length(vec))
  for(i in 1:length(vec)){
  	new[i] <- my_sum(vec[1:i])
  }
  new
}
    \end{verbatim}
  \end{quote}


\item Select the number in a vector that is the greatest distance from the first element of the vector
  \begin{quote}
    \begin{verbatim}
select <- function(vec){
  distance <- abs(vec - vec[1])
  vec[which(distance == max(distance))]
}
    \end{verbatim}
  \end{quote}

\item Return whether a vector of numbers is right skewed or left skewed by comparing its mean and median.
  \begin{quote}
    \begin{verbatim}
# note: this concept only works for large amounts of skew - Garrett
skew <- function(vec){
  if (mean(vec) < median(vec))
    return("left-skewed")
  if (mean(vec) > median(vec))
    return("right-skewed")
  return("symmetric")
}
    \end{verbatim}
  \end{quote}

\item Find the (statistical) mode of a vector.
  \begin{quote}
    \begin{verbatim}
mode_vec <- function(vec){
  counts <- as.data.frame(table(vec))$Freq
  vec[which(counts == max(counts))]
}
    \end{verbatim}
  \end{quote}

\item Given a numeric vector of length 100, determine which element occurs at the 70th percentile.
  \begin{quote}
    \begin{verbatim}
ptile100 <- function(vec){
  vec[order(vec)]
  vec[70]
}

    \end{verbatim}
  \end{quote}

\item Given a numeric vector of length 10, determine which element occurs at the 70th percentile.
  \begin{quote}
    \begin{verbatim}
ptile10 <- function(vec){
  vec[order(vec)]
  vec[7]
}

    \end{verbatim}
  \end{quote}

\item Given a numeric vector of length 10, determine which element occurs at the 70th percentile.
  \begin{quote}
    \begin{verbatim}
ptile <- function(vec){
  vec[order(vec)]
  vec[round(.7 * length(vec), 1)]
}

    \end{verbatim}
  \end{quote}


\item Return a vector with its elements reordered in a random manner.
  \begin{quote}
    \begin{verbatim}
shuffle <- function(vec){
  sample(vec, length(vec), replace = F) 
}
    \end{verbatim}
  \end{quote}

\item Return a vector with its elements ordered from smallest to largest.
  \begin{quote}
    \begin{verbatim}
order1 <- function(vec){
  vec[order(vec)]
}
    \end{verbatim}
  \end{quote}

\item Return a vector with its elements ordered largest to smallest.
  \begin{quote}
    \begin{verbatim}
order2 <- function(vec){
  vec[order(-vec)]
}
    \end{verbatim}
  \end{quote}





\end{enumerate}


\end{document}
